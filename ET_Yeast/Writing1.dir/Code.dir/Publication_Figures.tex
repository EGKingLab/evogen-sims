% Options for packages loaded elsewhere
\PassOptionsToPackage{unicode}{hyperref}
\PassOptionsToPackage{hyphens}{url}
\PassOptionsToPackage{dvipsnames,svgnames,x11names}{xcolor}
%
\documentclass[
  letterpaper,
  DIV=11,
  numbers=noendperiod]{scrartcl}
\usepackage{amsmath,amssymb}
\usepackage{lmodern}
\usepackage{iftex}
\ifPDFTeX
  \usepackage[T1]{fontenc}
  \usepackage[utf8]{inputenc}
  \usepackage{textcomp} % provide euro and other symbols
\else % if luatex or xetex
  \usepackage{unicode-math}
  \defaultfontfeatures{Scale=MatchLowercase}
  \defaultfontfeatures[\rmfamily]{Ligatures=TeX,Scale=1}
\fi
% Use upquote if available, for straight quotes in verbatim environments
\IfFileExists{upquote.sty}{\usepackage{upquote}}{}
\IfFileExists{microtype.sty}{% use microtype if available
  \usepackage[]{microtype}
  \UseMicrotypeSet[protrusion]{basicmath} % disable protrusion for tt fonts
}{}
\makeatletter
\@ifundefined{KOMAClassName}{% if non-KOMA class
  \IfFileExists{parskip.sty}{%
    \usepackage{parskip}
  }{% else
    \setlength{\parindent}{0pt}
    \setlength{\parskip}{6pt plus 2pt minus 1pt}}
}{% if KOMA class
  \KOMAoptions{parskip=half}}
\makeatother
\usepackage{xcolor}
\IfFileExists{xurl.sty}{\usepackage{xurl}}{} % add URL line breaks if available
\IfFileExists{bookmark.sty}{\usepackage{bookmark}}{\usepackage{hyperref}}
\hypersetup{
  pdftitle={Publication\_Figures},
  pdfauthor={ET},
  colorlinks=true,
  linkcolor={blue},
  filecolor={Maroon},
  citecolor={Blue},
  urlcolor={Blue},
  pdfcreator={LaTeX via pandoc}}
\urlstyle{same} % disable monospaced font for URLs
\usepackage{color}
\usepackage{fancyvrb}
\newcommand{\VerbBar}{|}
\newcommand{\VERB}{\Verb[commandchars=\\\{\}]}
\DefineVerbatimEnvironment{Highlighting}{Verbatim}{commandchars=\\\{\}}
% Add ',fontsize=\small' for more characters per line
\usepackage{framed}
\definecolor{shadecolor}{RGB}{241,243,245}
\newenvironment{Shaded}{\begin{snugshade}}{\end{snugshade}}
\newcommand{\AlertTok}[1]{\textcolor[rgb]{0.68,0.00,0.00}{#1}}
\newcommand{\AnnotationTok}[1]{\textcolor[rgb]{0.37,0.37,0.37}{#1}}
\newcommand{\AttributeTok}[1]{\textcolor[rgb]{0.40,0.46,0.14}{#1}}
\newcommand{\BaseNTok}[1]{\textcolor[rgb]{0.68,0.00,0.00}{#1}}
\newcommand{\BuiltInTok}[1]{\textcolor[rgb]{0.00,0.46,0.62}{#1}}
\newcommand{\CharTok}[1]{\textcolor[rgb]{0.13,0.47,0.30}{#1}}
\newcommand{\CommentTok}[1]{\textcolor[rgb]{0.37,0.37,0.37}{#1}}
\newcommand{\CommentVarTok}[1]{\textcolor[rgb]{0.37,0.37,0.37}{\textit{#1}}}
\newcommand{\ConstantTok}[1]{\textcolor[rgb]{0.56,0.35,0.01}{#1}}
\newcommand{\ControlFlowTok}[1]{\textcolor[rgb]{0.00,0.46,0.62}{#1}}
\newcommand{\DataTypeTok}[1]{\textcolor[rgb]{0.68,0.00,0.00}{#1}}
\newcommand{\DecValTok}[1]{\textcolor[rgb]{0.68,0.00,0.00}{#1}}
\newcommand{\DocumentationTok}[1]{\textcolor[rgb]{0.37,0.37,0.37}{\textit{#1}}}
\newcommand{\ErrorTok}[1]{\textcolor[rgb]{0.68,0.00,0.00}{#1}}
\newcommand{\ExtensionTok}[1]{\textcolor[rgb]{0.00,0.46,0.62}{#1}}
\newcommand{\FloatTok}[1]{\textcolor[rgb]{0.68,0.00,0.00}{#1}}
\newcommand{\FunctionTok}[1]{\textcolor[rgb]{0.28,0.35,0.67}{#1}}
\newcommand{\ImportTok}[1]{\textcolor[rgb]{0.00,0.46,0.62}{#1}}
\newcommand{\InformationTok}[1]{\textcolor[rgb]{0.37,0.37,0.37}{#1}}
\newcommand{\KeywordTok}[1]{\textcolor[rgb]{0.00,0.46,0.62}{#1}}
\newcommand{\NormalTok}[1]{\textcolor[rgb]{0.00,0.46,0.62}{#1}}
\newcommand{\OperatorTok}[1]{\textcolor[rgb]{0.37,0.37,0.37}{#1}}
\newcommand{\OtherTok}[1]{\textcolor[rgb]{0.00,0.46,0.62}{#1}}
\newcommand{\PreprocessorTok}[1]{\textcolor[rgb]{0.68,0.00,0.00}{#1}}
\newcommand{\RegionMarkerTok}[1]{\textcolor[rgb]{0.00,0.46,0.62}{#1}}
\newcommand{\SpecialCharTok}[1]{\textcolor[rgb]{0.37,0.37,0.37}{#1}}
\newcommand{\SpecialStringTok}[1]{\textcolor[rgb]{0.13,0.47,0.30}{#1}}
\newcommand{\StringTok}[1]{\textcolor[rgb]{0.13,0.47,0.30}{#1}}
\newcommand{\VariableTok}[1]{\textcolor[rgb]{0.07,0.07,0.07}{#1}}
\newcommand{\VerbatimStringTok}[1]{\textcolor[rgb]{0.13,0.47,0.30}{#1}}
\newcommand{\WarningTok}[1]{\textcolor[rgb]{0.37,0.37,0.37}{\textit{#1}}}
\usepackage{longtable,booktabs,array}
\usepackage{calc} % for calculating minipage widths
% Correct order of tables after \paragraph or \subparagraph
\usepackage{etoolbox}
\makeatletter
\patchcmd\longtable{\par}{\if@noskipsec\mbox{}\fi\par}{}{}
\makeatother
% Allow footnotes in longtable head/foot
\IfFileExists{footnotehyper.sty}{\usepackage{footnotehyper}}{\usepackage{footnote}}
\makesavenoteenv{longtable}
\usepackage{graphicx}
\makeatletter
\def\maxwidth{\ifdim\Gin@nat@width>\linewidth\linewidth\else\Gin@nat@width\fi}
\def\maxheight{\ifdim\Gin@nat@height>\textheight\textheight\else\Gin@nat@height\fi}
\makeatother
% Scale images if necessary, so that they will not overflow the page
% margins by default, and it is still possible to overwrite the defaults
% using explicit options in \includegraphics[width, height, ...]{}
\setkeys{Gin}{width=\maxwidth,height=\maxheight,keepaspectratio}
% Set default figure placement to htbp
\makeatletter
\def\fps@figure{htbp}
\makeatother
\setlength{\emergencystretch}{3em} % prevent overfull lines
\providecommand{\tightlist}{%
  \setlength{\itemsep}{0pt}\setlength{\parskip}{0pt}}
\setcounter{secnumdepth}{-\maxdimen} % remove section numbering
\KOMAoption{captions}{tableheading}
\makeatletter
\makeatother
\makeatletter
\@ifpackageloaded{caption}{}{\usepackage{caption}}
\AtBeginDocument{%
\renewcommand*\contentsname{Table of contents}
\renewcommand*\listfigurename{List of Figures}
\renewcommand*\listtablename{List of Tables}
\renewcommand*\figurename{Figure}
\renewcommand*\tablename{Table}
}
\@ifpackageloaded{float}{}{\usepackage{float}}
\floatstyle{ruled}
\@ifundefined{c@chapter}{\newfloat{codelisting}{h}{lop}}{\newfloat{codelisting}{h}{lop}[chapter]}
\floatname{codelisting}{Listing}
\newcommand*\listoflistings{\listof{codelisting}{List of Listings}}
\makeatother
\makeatletter
\@ifpackageloaded{caption}{}{\usepackage{caption}}
\@ifpackageloaded{subcaption}{}{\usepackage{subcaption}}
\makeatother
\makeatletter
\makeatother
\ifLuaTeX
  \usepackage{selnolig}  % disable illegal ligatures
\fi

\title{Publication\_Figures}
\author{ET}
\date{}

\begin{document}
\maketitle

\hypertarget{allele-frequency}{%
\subsection{Allele Frequency}\label{allele-frequency}}

\begin{verbatim}
Attaching package: 'dplyr'
\end{verbatim}

\begin{verbatim}
The following objects are masked from 'package:stats':

    filter, lag
\end{verbatim}

\begin{verbatim}
The following objects are masked from 'package:base':

    intersect, setdiff, setequal, union
\end{verbatim}

\begin{verbatim}
Attaching package: 'cowplot'
\end{verbatim}

\begin{verbatim}
The following object is masked from 'package:patchwork':

    align_plots
\end{verbatim}

\begin{verbatim}
Loading required package: foreach
\end{verbatim}

\begin{verbatim}
Attaching package: 'foreach'
\end{verbatim}

\begin{verbatim}
The following objects are masked from 'package:purrr':

    accumulate, when
\end{verbatim}

\begin{verbatim}
Loading required package: iterators
\end{verbatim}

\begin{verbatim}
Loading required package: parallel
\end{verbatim}

\hypertarget{linear-polygenic-models}{%
\subsection{Linear Polygenic Models}\label{linear-polygenic-models}}

\hypertarget{small-vs-high-parameters}{%
\subsubsection{Small vs High
parameters}\label{small-vs-high-parameters}}

\includegraphics{Publication_Figures_files/figure-pdf/unnamed-chunk-2-1.pdf}

\hypertarget{heritability-effect}{%
\subsubsection{Heritability Effect}\label{heritability-effect}}

\includegraphics{Publication_Figures_files/figure-pdf/unnamed-chunk-3-1.pdf}

\hypertarget{heritability-and-selection-pressure}{%
\subsubsection{Heritability and Selection
Pressure}\label{heritability-and-selection-pressure}}

\includegraphics{Publication_Figures_files/figure-pdf/unnamed-chunk-4-1.pdf}

\hypertarget{selection-length-effect}{%
\subsubsection{Selection Length Effect}\label{selection-length-effect}}

\includegraphics{Publication_Figures_files/figure-pdf/unnamed-chunk-5-1.pdf}

\hypertarget{selection-length-vs-heritability-effect}{%
\subsubsection{Selection Length vs Heritability
Effect}\label{selection-length-vs-heritability-effect}}

\includegraphics{Publication_Figures_files/figure-pdf/unnamed-chunk-6-1.pdf}

\hypertarget{selection-length-vs-selection-strength-effect}{%
\subsubsection{Selection Length vs Selection Strength
Effect}\label{selection-length-vs-selection-strength-effect}}

\includegraphics{Publication_Figures_files/figure-pdf/unnamed-chunk-7-1.pdf}

\hypertarget{all-in-one}{%
\subsection{All in one}\label{all-in-one}}

\includegraphics{Publication_Figures_files/figure-pdf/unnamed-chunk-8-1.pdf}

\#\#\#\#Let's try frequency with Effect

\includegraphics{Publication_Figures_files/figure-pdf/unnamed-chunk-11-1.pdf}

\hypertarget{effect-of-qtl-and-initial-frequency}{%
\subsubsection{Effect of QTL and Initial
Frequency}\label{effect-of-qtl-and-initial-frequency}}

\includegraphics{Publication_Figures_files/figure-pdf/unnamed-chunk-12-1.pdf}

\includegraphics{Publication_Figures_files/figure-pdf/unnamed-chunk-13-1.pdf}

\includegraphics{Publication_Figures_files/figure-pdf/unnamed-chunk-14-1.pdf}

\hypertarget{many-replicates}{%
\subsubsection{Many replicates}\label{many-replicates}}

\includegraphics{Publication_Figures_files/figure-pdf/unnamed-chunk-15-1.pdf}

\includegraphics{Publication_Figures_files/figure-pdf/unnamed-chunk-15-2.pdf}

\includegraphics{Publication_Figures_files/figure-pdf/unnamed-chunk-15-3.pdf}

\includegraphics{Publication_Figures_files/figure-pdf/unnamed-chunk-15-4.pdf}

\includegraphics{Publication_Figures_files/figure-pdf/unnamed-chunk-15-5.pdf}

\includegraphics{Publication_Figures_files/figure-pdf/unnamed-chunk-15-6.pdf}

\includegraphics{Publication_Figures_files/figure-pdf/unnamed-chunk-15-7.pdf}

\includegraphics{Publication_Figures_files/figure-pdf/unnamed-chunk-15-8.pdf}

\includegraphics{Publication_Figures_files/figure-pdf/unnamed-chunk-16-1.pdf}

\includegraphics{Publication_Figures_files/figure-pdf/unnamed-chunk-16-2.pdf}

\includegraphics{Publication_Figures_files/figure-pdf/unnamed-chunk-16-3.pdf}

\includegraphics{Publication_Figures_files/figure-pdf/unnamed-chunk-16-4.pdf}

\includegraphics{Publication_Figures_files/figure-pdf/unnamed-chunk-16-5.pdf}

\includegraphics{Publication_Figures_files/figure-pdf/unnamed-chunk-16-6.pdf}

\includegraphics{Publication_Figures_files/figure-pdf/unnamed-chunk-16-7.pdf}

\includegraphics{Publication_Figures_files/figure-pdf/unnamed-chunk-16-8.pdf}

\hypertarget{phenotypes}{%
\subsubsection{Phenotypes}\label{phenotypes}}

\includegraphics{Publication_Figures_files/figure-pdf/unnamed-chunk-17-1.pdf}

\includegraphics{Publication_Figures_files/figure-pdf/unnamed-chunk-18-1.pdf}

\hypertarget{monogenic-model}{%
\paragraph{Monogenic model}\label{monogenic-model}}

\includegraphics{Publication_Figures_files/figure-pdf/unnamed-chunk-19-1.pdf}

\includegraphics{Publication_Figures_files/figure-pdf/unnamed-chunk-21-1.pdf}

\hypertarget{oligogenic-model}{%
\paragraph{Oligogenic model}\label{oligogenic-model}}

\includegraphics{Publication_Figures_files/figure-pdf/unnamed-chunk-22-1.pdf}

\includegraphics{Publication_Figures_files/figure-pdf/unnamed-chunk-24-1.pdf}

\hypertarget{section}{%
\subsection{}\label{section}}

\hypertarget{comparisons}{%
\subsection{Comparisons}\label{comparisons}}

\hypertarget{figure-1-monogenic-two-season}{%
\subsubsection{Figure 1: Monogenic
Two-Season}\label{figure-1-monogenic-two-season}}

\hypertarget{figure-2-models-comparison}{%
\subsubsection{Figure 2: Models
Comparison}\label{figure-2-models-comparison}}

\begin{Shaded}
\begin{Highlighting}[]
\CommentTok{\# Add a column to distinguish the datasets}
\NormalTok{geno\_file }\OtherTok{\textless{}{-}}\NormalTok{ mono1 }\SpecialCharTok{\%\textgreater{}\%} \FunctionTok{mutate}\NormalTok{(}\AttributeTok{Selection =} \StringTok{"Instantaneous Model"}\NormalTok{)}
\NormalTok{geno\_file2 }\OtherTok{\textless{}{-}}\NormalTok{ mono2 }\SpecialCharTok{\%\textgreater{}\%} \FunctionTok{mutate}\NormalTok{(}\AttributeTok{Selection =} \StringTok{"Gradual 2{-}Seasons Model"}\NormalTok{)}
\NormalTok{geno\_file3 }\OtherTok{\textless{}{-}}\NormalTok{ mono3 }\SpecialCharTok{\%\textgreater{}\%} \FunctionTok{mutate}\NormalTok{(}\AttributeTok{Selection =} \StringTok{"Gradual 4{-}Seasons Model"}\NormalTok{)}

\CommentTok{\# Combine the datasets}
\NormalTok{Genome\_dataset }\OtherTok{\textless{}{-}} \FunctionTok{rbind}\NormalTok{(geno\_file, }
\NormalTok{                        geno\_file2, geno\_file3) }\SpecialCharTok{\%\textgreater{}\%} 
  \FunctionTok{mutate}\NormalTok{(}\AttributeTok{Selection =} \FunctionTok{fct\_relevel}\NormalTok{(Selection, }
                                 \StringTok{"Instantaneous Model"}\NormalTok{, }
                                 \StringTok{"Gradual 2{-}Seasons Model"}\NormalTok{, }
                                 \StringTok{"Gradual 4{-}Seasons Model"}\NormalTok{))}

\CommentTok{\# Create the plot}
\NormalTok{freqplot }\OtherTok{\textless{}{-}}\NormalTok{ Genome\_dataset }\SpecialCharTok{\%\textgreater{}\%} 
  \FunctionTok{filter}\NormalTok{(Generation }\SpecialCharTok{\textless{}=} \DecValTok{500}\NormalTok{) }\SpecialCharTok{\%\textgreater{}\%} 
  \FunctionTok{ggplot}\NormalTok{(}\FunctionTok{aes}\NormalTok{(}\AttributeTok{x =}\NormalTok{ Generation, }\AttributeTok{y =}\NormalTok{ Frequency, }\AttributeTok{color =} \FunctionTok{factor}\NormalTok{(Position))) }\SpecialCharTok{+}
  \FunctionTok{geom\_line}\NormalTok{(}\AttributeTok{size =} \FloatTok{0.5}\NormalTok{) }\SpecialCharTok{+}
  \FunctionTok{facet\_wrap}\NormalTok{(Selection}\SpecialCharTok{\textasciitilde{}}\NormalTok{.) }\SpecialCharTok{+}
  \FunctionTok{theme\_cowplot}\NormalTok{()}\SpecialCharTok{+}
  \FunctionTok{theme}\NormalTok{(}\AttributeTok{legend.position =} \StringTok{"none"}\NormalTok{,}
        \AttributeTok{axis.text =} \FunctionTok{element\_text}\NormalTok{( }\AttributeTok{size =} \DecValTok{15}\NormalTok{, }\AttributeTok{face =} \StringTok{"bold"}\NormalTok{),}
        \AttributeTok{axis.line =} \FunctionTok{element\_line}\NormalTok{(}\AttributeTok{size =} \DecValTok{3}\NormalTok{),}
        \AttributeTok{axis.title =} \FunctionTok{element\_text}\NormalTok{( }\AttributeTok{size =} \DecValTok{15}\NormalTok{, }\AttributeTok{face =} \StringTok{"bold"}\NormalTok{))}

\NormalTok{freqplot}
\end{Highlighting}
\end{Shaded}

\begin{figure}[H]

{\centering \includegraphics{Publication_Figures_files/figure-pdf/unnamed-chunk-27-1.pdf}

}

\end{figure}

\hypertarget{neutral-vs-gradual-4-seasons}{%
\subsubsection{Neutral vs Gradual 4
seasons}\label{neutral-vs-gradual-4-seasons}}

\includegraphics{Publication_Figures_files/figure-pdf/unnamed-chunk-29-1.pdf}

\hypertarget{genome-size}{%
\subsubsection{Genome Size}\label{genome-size}}

\includegraphics{Publication_Figures_files/figure-pdf/unnamed-chunk-31-1.pdf}

\hypertarget{gradual-i}{%
\subsubsection{Gradual I}\label{gradual-i}}

\hypertarget{polygenic}{%
\paragraph{Polygenic}\label{polygenic}}

\includegraphics{Publication_Figures_files/figure-pdf/unnamed-chunk-32-1.pdf}

\hypertarget{oligogenic}{%
\paragraph{Oligogenic}\label{oligogenic}}

\includegraphics{Publication_Figures_files/figure-pdf/unnamed-chunk-33-1.pdf}

\hypertarget{monogenic}{%
\paragraph{Monogenic}\label{monogenic}}

\includegraphics{Publication_Figures_files/figure-pdf/unnamed-chunk-34-1.pdf}

\hypertarget{gradual-ii}{%
\subsection{Gradual II}\label{gradual-ii}}

\hypertarget{allele-frequency-1}{%
\subsubsection{Allele Frequency}\label{allele-frequency-1}}

\hypertarget{polygenic-1}{%
\paragraph{Polygenic}\label{polygenic-1}}

\includegraphics{Publication_Figures_files/figure-pdf/unnamed-chunk-35-1.pdf}

\hypertarget{oligogenic-1}{%
\paragraph{Oligogenic}\label{oligogenic-1}}

\includegraphics{Publication_Figures_files/figure-pdf/unnamed-chunk-36-1.pdf}

\hypertarget{monogenic-1}{%
\paragraph{Monogenic}\label{monogenic-1}}

\includegraphics{Publication_Figures_files/figure-pdf/unnamed-chunk-37-1.pdf}

\hypertarget{phenotypes-1}{%
\subsubsection{Phenotypes}\label{phenotypes-1}}

\hypertarget{polygenic-2}{%
\paragraph{Polygenic}\label{polygenic-2}}

\includegraphics{Publication_Figures_files/figure-pdf/unnamed-chunk-38-1.pdf}

\hypertarget{oligogenic-2}{%
\paragraph{Oligogenic}\label{oligogenic-2}}

\includegraphics{Publication_Figures_files/figure-pdf/unnamed-chunk-39-1.pdf}

\hypertarget{monogenic-2}{%
\paragraph{Monogenic}\label{monogenic-2}}

\includegraphics{Publication_Figures_files/figure-pdf/unnamed-chunk-40-1.pdf}

\hypertarget{constant}{%
\subsection{Constant}\label{constant}}

\hypertarget{allele-frequency-2}{%
\subsubsection{Allele Frequency}\label{allele-frequency-2}}

\hypertarget{polygenic-3}{%
\paragraph{Polygenic}\label{polygenic-3}}

\includegraphics{Publication_Figures_files/figure-pdf/unnamed-chunk-41-1.pdf}

\hypertarget{oligogenic-3}{%
\paragraph{Oligogenic}\label{oligogenic-3}}

\includegraphics{Publication_Figures_files/figure-pdf/unnamed-chunk-42-1.pdf}

\hypertarget{monogenic-3}{%
\paragraph{Monogenic}\label{monogenic-3}}

\includegraphics{Publication_Figures_files/figure-pdf/unnamed-chunk-43-1.pdf}

\hypertarget{phenotypes-2}{%
\subsubsection{Phenotypes}\label{phenotypes-2}}

\hypertarget{polygenic-4}{%
\paragraph{Polygenic}\label{polygenic-4}}

\includegraphics{Publication_Figures_files/figure-pdf/unnamed-chunk-44-1.pdf}

\hypertarget{oligogenic-4}{%
\paragraph{Oligogenic}\label{oligogenic-4}}

\includegraphics{Publication_Figures_files/figure-pdf/unnamed-chunk-45-1.pdf}

\hypertarget{monogenic-4}{%
\paragraph{Monogenic}\label{monogenic-4}}

\includegraphics{Publication_Figures_files/figure-pdf/unnamed-chunk-46-1.pdf}

\hypertarget{neutral}{%
\subsection{Neutral}\label{neutral}}

\hypertarget{allele-frequency-3}{%
\subsubsection{Allele Frequency}\label{allele-frequency-3}}

\hypertarget{polygenic-5}{%
\paragraph{Polygenic}\label{polygenic-5}}

\includegraphics{Publication_Figures_files/figure-pdf/unnamed-chunk-47-1.pdf}

\hypertarget{oligogenic-5}{%
\paragraph{Oligogenic}\label{oligogenic-5}}

\includegraphics{Publication_Figures_files/figure-pdf/unnamed-chunk-48-1.pdf}

\hypertarget{monogenic-5}{%
\paragraph{Monogenic}\label{monogenic-5}}

\includegraphics{Publication_Figures_files/figure-pdf/unnamed-chunk-49-1.pdf}

\hypertarget{phenotypes-3}{%
\subsubsection{Phenotypes}\label{phenotypes-3}}

\hypertarget{polygenic-6}{%
\paragraph{Polygenic}\label{polygenic-6}}

\includegraphics{Publication_Figures_files/figure-pdf/unnamed-chunk-50-1.pdf}

\hypertarget{oligogenic-6}{%
\paragraph{Oligogenic}\label{oligogenic-6}}

\includegraphics{Publication_Figures_files/figure-pdf/unnamed-chunk-51-1.pdf}

\hypertarget{monogenic-6}{%
\paragraph{Monogenic}\label{monogenic-6}}

\includegraphics{Publication_Figures_files/figure-pdf/unnamed-chunk-52-1.pdf}

\end{document}
